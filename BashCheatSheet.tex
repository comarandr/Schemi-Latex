\documentclass{article}
\usepackage{graphicx} % Required for inserting images
\usepackage{amssymb}


\title{Bash cheat sheet}
\author{Andrea}
\begin{document}

\maketitle

\newpage

\section{Comandi Introduttivi Bash}
\begin{itemize} %inizio elenco sezioni

    \item Tipi di shell
    
        \begin{itemize} %inizio elenco comandi shell
        
        \item \textbf{sh} : cambia in Bourne shell
        \item \textbf{bash} :  cambia in Bourne again shell
        \item \textbf{csh}, \textbf{tcs} : ulteriori tipi
        \item \textbf{echo \$Shell} : mostra tipo bash
        \item \textbf{exec} : tipo di shell
        \item \textbf{man [arg] [comando]} : apre il manuale alla pagina del comando
            \begin{itemize}
                \item \textbf{-a} : apre nelle sezioni in cui compare comando
                \item $\mathbf{ -s_{i} }$ : apre nella i-esima sezione $(s_{i} \ intero)$
            \end{itemize}
        \item \textbf{clear} : ripulisce la schermata
        \item \textbf{exit} , \textbf{CTRL+T} , \textbf{logout} : chiusura shell
        
        \end{itemize} %fine elenco comandi shell
        
    \item File System
    
        \begin{itemize} %inizio gestione cartelle
        
            \item \textbf{pwd } : cartella in cui ti trovi
            \item \textbf{ cd [path] } : cambia cartella
            \item \textbf{..} : cartella madre
            \item \textbf{.} : cartella corrente
            \item \textbf{ls [arg] [path] } : elenca i file
            
            \begin{itemize} %inizio argomenti ls
            
                \item \textbf{-l} : elenco lungo
                \item \textbf{-a} : elenca elementi nascosti
                \item \textbf{-al} : elenco lungo e nascosti, posso combinare elementi 
                
            \end{itemize} %fine argomenti ls
            
            \item \textbf{mkdir} : crea cartella
            \item \textbf{rmdir} : rimuove cartella
            \item \textbf{rm [option] [file]} ; rimuove file o cartella
            \item \textbf{cp [file1] [file2]} : copia file1 in fil2, se file2 non esiste lo crea
            \item \textbf{cp [file1] [file2] [...] [dir1]} : copia più file, ma ultimo argomento cartella
            \item \textbf{mv [file1] [file2] } : sposta, rimane solo file2
            \item \textbf{mv [file1] [file2] [...] [dir1]} : sposta più file, ultimo argomento cartella
            \item \textbf{ln [file1] [file2]} : creazione di hard link
            \item \textbf{ln $\mathbf{-s}$ [file1] [file2] }: creazione di soft link
            
        \end{itemize} %fine gestione cartelle
        \newpage
        
    \item Permessi ( rwx: read, write, execute) 
        \begin{itemize} %inizio permessi
            \item \textbf{chmod [arg] [file1]} : cambia permessi 
                \begin{itemize} %inizio argomenti chmod
                    \item sistema ottale, es 744
                    \item u=rwx : owner
                    \item g=rwx : gruppo
                    \item o=rwx : altri utenti
                    \item + : aggiunge permessi che seguono
                    \item - : toglie permessi che seguono
                    \item = : imposta esattamente permessi che seguono
                \end{itemize} %fine argomenti chmod
        \end{itemize} %fine permessi
    
    \item Comandi File
    \begin{itemize} %inizio comandi file
     
    
    \item \textbf{cat [file1]} : mostra contenuto file tutto in una volta (oppure concatena)
    \item \textbf{more [file1]} : mostra contenuto file ma una pagina alla volta
    \item \textbf{less [file1]} : mostra contenuto file una pagina alla volta ma puoi scorrere
    \item \textbf{tail  [file1]} : mostra ultime righe contenuto file, posso specificare con -n (n $\in \mathbb{N}$)
    \item \textbf{head [file1]} : mostra le prime righe di un file, posso specificare con -n (n $\in \mathbb{N}$)
    \item \textbf{wc [file1]} : conta righe, parole e caratteri
    \end{itemize} %fine comandi file

    \item Metacaratteri
        \begin{itemize} %inizio metacaretteri
            \item $\mathbf{\ast}$ : abbreviazione che indica generica stringa di 0 o più caratteri
            \item \textbf{?} : abbreviazione per singolo carattere generico
            \item \textbf{[]} : singolo carattere tra quelli elencati
            \item \{\} : sequenza di stringhe
            \item $>$ : redirezione output
            \item $<<$: redirezione output (append)
            \item $<$ : redirezione dell'input
            \item $<<$ : redirezione input dalla linea comando
            \item $2>$ : redirezione messaggi errore
            \item $|$ : pipe, compone comandi in parallelo
            \item ; : sequenza di comandi
            \item $||$ : se primo fallisce, esegue successivo
            \item \&\& : se primo comando riesce, esegue successivo
            \item (): raggruppamento comandi
            \item $\ast$ : introduce un commento
            \item ! : ripetizione di comandi memorizzati history list
            

            \item \textbf{quoting} : inibisce l'effetto dei metacaratteri

                \begin{itemize} %inizio quoting
                    \item \textbf{' [comando] ' }
                    \item \textbf{\textbackslash [metachar]}
                    \item \textbf{" [comando] "}
                \end{itemize} %fine quoting
        \end{itemize} %finecmetacaratteri


    \item History e comandi
        \begin{itemize} %inizio history
            \item \textbf{history} : visualizza lista comandi
            \item \textbf{alias [arg] [comando]} : crea alias
                \begin{itemize}
                    \item \textbf{.} : alias cartella corrente
                    \item \textbf{..} : alias cartella madre
                    \item \textbf{~username} : alias home directory
                \end{itemize}
            \item \textbf{unalias [arg] [comando]} : rimuove alias
        \end{itemize} %fine history

    \item Processi
        \begin{itemize} %inizio comandi processi
            \item \textbf{ps [arg]} : fornisce processi dell'utente associati al terminale corrente
            \begin{itemize} %inizio argomenti ps
                \item -a : tutti processi associati a un terminale
                \item -f : full listing
                \item -e : tutti processi anche non associati a un terminale
                \item -l : long listing
            \end{itemize} %fine argomenti ps

            \item \textbf{kill [arg] [processo]} : arresta il processo
            \begin{itemize} %inizio argomenti kill
                \item -9 : segnale sigkill
                \item -s kill : segnale sigkill
            \end{itemize} %fine argomenti kill
            \item \textbf{[comando] \&} : esegue in background
            \item \textbf{jobs} : mostra lista dei job in esecuzione
            \item \textbf{CTRL+Z} : sospende un job
            \item \textbf{kill \%i} : termina il job di indice i
            \item \textbf{[job] fg} : resume del job in foreground
            \item \textbf{[job] bg} : resume del job in background
            
        \end{itemize} %fine comandi processi
\end{itemize} %fine elenco sezioni


\end{document}
