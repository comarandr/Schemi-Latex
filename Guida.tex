\documentclass{article} %tipo di documento
\usepackage[utf8]{inputenc}
\usepackage[T1]{fontenc}
\usepackage{algorithm2e}
\usepackage{graphicx} % Required for inserting images
%before \begin
\usepackage{amssymb} %per i simboli matematici
\usepackage{mathtools} %per simboli 
\usepackage{listings} %per scrivere codice
\usepackage{color} %per colorare il codice
\usepackage{verbatim} %per scrivere codice
\DeclarePairedDelimiter\ceil{\lceil}{\rceil}
\DeclarePairedDelimiter\floor{\lfloor}{\rfloor}


\title{Prova di comandi Latex}
\author{Andrea Comar}
\date{October 2024}

\begin{document}

\maketitle % to show title,author,date

\tableofcontents % to show table of contents

\newpage

\part{ Titolo parte }
Questo è del testo a caso per far capire cosa succede se scrivi direttamente sotto una parte.
La parte, con comando \textbackslash part è la sezione più grossa in Latex. In automatico scrive parte I sopra. 

\section{ Titolo sezione }
La sezione è la seconda parte e puoi vedere il font automatico sopra queste righe di testo.
comando \textbackslash section
\subsection{ subsection }
La sottosezione è questa parte, comando \textbackslash subsection
\section{ Titolo sezione }
Riutilizzando section conta in automatico quante sono le sezioni nel documento. Vale per tutti i comandi.
\subsection{ titolo sottosezione}



\paragraph{titolo paragrafo}
Questo è un esempio di testo all'interno del paragrafo. 
Gli spazi sono automatici, non c'è bisogno di premere invio per andare a capo.
\subparagraph{titolo sottoparagrafo }  In questo caso invece sto scrivendo un sottoparagrafo del precedente. 
Nel caso successivo provo a scrivere direttamente.
\subparagraph{secondo sottoparagrafo} questo è un ulteriore sottoparagrafo.

Se vuoi cambiare pagina, devi usare \textbackslash newpage
\newpage


\part{ Cose utili } %inizio parte con i comandi utili
\section{Nuovo documento} 
Per creare un nuovo documento, basta scrivere:
\begin{verbatim}
\documentclass{article} % tipo di documento
	\usepackage{nome_pacchetto} % per i pacchetti

	\title{titolo} 
	\author{autore}
	\date{data}

\begin{document} % inizio del documento

	\maketitle % per mostrare titolo, autore, data
	\tableofcontents % per mostrare la tabella dei contenuti
	...
	\newpage % per cambiare pagina
	...
\end{document} 
\end{verbatim}

La sintassi minima è la seguente
\begin{verbatim}
\documentclass{article} 

\begin{document}
	...
\end{document}
\end{verbatim}

\section{pacchetti utili} %PARTE DEI PACCHETTI

\paragraph{graphicx} per inserire immagini
\paragraph{amssymb} per i simboli matematici
\paragraph{mathtools} per i simboli matematici
\paragraph{algorithm2e} per scrivere algoritmi
\paragraph{listings} per scrivere codice. si può scegliere un linguaggio oppure usarlo in modo simile al comando \textbackslash verbatim. Consiglio la seguente impostazione:
\begin{verbatim}
	\lstset{
  basicstyle=\ttfamily,
  mathescape
}
\end{verbatim} 
permette di scrivere i simboli matematici all'interno del codice, racchiundendoli con \$ ... \$
\paragraph{color} per colorare il codice

\section{Comandi scrittura} %PARTE DEI COMANDI DI SCRITTURA

\paragraph{\textbackslash verbatin} Permette a Latex di non compilare il 
testo all'interno delle parentesi graffe.

\begin{verbatim}
	if (a > b) {
		return a;
	} else {
		return b;
	}
\end{verbatim}

\paragraph{\textbackslash begin\{tabbing\}}
Permette di scrivere all'interno di una tabulazione.
uso dello \textbackslash \ per differenziare colonne.

\begin{verbatim}
	\begin{tabbing}
			ciao\ ciao \ ciao
	\end{tabbing}	
\end{verbatim}
\begin{tabbing}
ciao \ ciao \ ciao
\end{tabbing}

\paragraph{\textbackslash textcolor\{colore\}\{testo\}}
\textcolor{red}{testo rosso} $\sim$ necessita del rispettivo pacchetto.

\paragraph{\textbackslash textit\{testo\}}
\textit{testo in corsivo}

\paragraph{\textbackslash textbf\{testo\}}
\textbf{testo in grassetto}

\paragraph{\textbackslash underline\{testo\}}
\underline{testo sottolineato}




\section{Utilizzo delle tabelle} %PARTE DELLE TABELLE
In questa sezione capiremo come creare tabelle su latex. proviamo!
\subsection{ambiente tabular}
L'ambiente tabular si chiama con
\begin{verbatim}
\begin{tabular}{condizioni}
	...
\end{tabular}
\end{verbatim}
Parametri:
\begin{itemize}
	\item parametri \textcolor{blue}{l,c,r} per allineamento orizzontale
	\item parametro \textcolor{blue}{p\{larghezza\}} per la larghezza. esclude parametri allineamento. 
	\item parametro \textcolor{blue}{$|$} per linee verticali
	\item paramatro \textcolor{blue}{@\{...\}} per spaziare le colonne
\end{itemize}
Costruzione tabella:
\begin{itemize}
	\item \textcolor{blue}{\&} per separare le colonne 
	\item \textcolor{blue}{\textbackslash\textbackslash} per andare a capo
	\item \textcolor{blue}{\textbackslash hline} inserisce linea orizzontale
	\item \textcolor{blue}{\textbackslash cline\{i-j\}} inserisce linea orizzontale tra le colonne i e j
	\item \textcolor{blue}{\textbackslash multicolumn\{ncol\}\{allineamentooriz\}\{...\}} per unire n colonne
\end{itemize}

\newpage

\subsection{Esempio tabella}
 %prima tabella
\begin{tabbing} 

\begin{tabular}{c}
\begin{lstlisting}
\begin{tabular}{|l|r|c|}
  \hline
  \multicolumn{3}{|c|}{all}\\
  \hline
  l & r & c \\
  \hline
\end{tabular}
\end{lstlisting}
\end{tabular}
\

\begin{tabular}{|l|r|c|}
	\hline
	\multicolumn{3}{|c|}{all}\\
	\hline
	l & r & c \\
	\hline
\end{tabular} 
\end{tabbing}

\begin{tabbing} %seconda tabella
\begin{tabular}{c}
\begin{lstlisting}
\begin{tabular}{|l|r|c|}
	\hline
	\multicolumn{3}{|c|}{allineamento}\\
	\hline
	l & r & c \\
	\hline
\end{tabular}
\end{lstlisting}
\end{tabular}
\

\begin{tabular}{|l|r|c|}
	\hline
	\multicolumn{3}{|c|}{allineamento}\\
	\hline
	l & r & c \\
	\hline
\end{tabular} 
\end{tabbing}
% close the tabbing environment

\begin{tabbing}%terza tabella

\begin{tabular}{c}
\begin{lstlisting}
\begin{tabular}{|c|l|r|}
	\hline
	\multicolumn{3}{|c|}{allineamento}\\
	\hline
	c & l & r \\	
	\hline
\end{tabular}
\end{lstlisting}
\end{tabular}

\

\begin{tabular}{|c|l|r|}
	\hline
	\multicolumn{3}{|c|}{allineamento}\\
	\hline
	c & l & r \\	
	\hline
\end{tabular} 
\end{tabbing}



 %quarta tabella

\begin{tabular}{c}
\begin{lstlisting}
\begin{tabular}{|r|r|r|}
	\hline
	\multicolumn{3}{|c|}{allineamento}\\
	\hline
	testo di prova & prova & interpretazione \\	
	\hline
\end{tabular}
\end{lstlisting}
\end{tabular}
\\
\begin{tabular}{|r|r|r|}
	\hline
	\multicolumn{3}{|c|}{allineamento}\\
	\hline
	testo di prova & prova & interpretazione \\	
	\hline
\end{tabular} 

\begin{tabular}{|l|r|c|}
	\hline
	\multicolumn{3}{|c|}{allineamento}\\
	\hline
	l & r & c \\
	\hline
\end{tabular}


\begin{tabular}{|c|l|r|}
	\hline
	\multicolumn{3}{|c|}{allineamento}\\
	\hline
	c & l & r \\	
	\hline
\end{tabular}




\begin{tabular}{|r|r|r|}
	\hline
	\multicolumn{3}{|c|}{allineamento}\\
	\hline
	testo di prova & prova & interpretazione \\	
	\hline
\end{tabular}





\begin{tabular}{|c|c|c|c|}
		\hline
		\multicolumn{4}{|c|}{Tabella di prova} \\
		\hline
		\textbf{Nome} & \textbf{Cognome} & \textbf{Età} & \textbf{Sesso} \\
		\hline
		Andrea & Comar & 23 & M \\
		\hline
		Stefano & Giorda & 23 & F \\
		\hline
\end{tabular}



\section{Simboli utili} %PARTE DEI SIMBOLI
\subsection{Simboli matematici}

\begin{tabular}{|c|c|c|c|}
	\hline
	\multicolumn{4}{|c|}{Simboli matematici} \\
	\hline
	\textbf{Simbolo} & \textbf{Comando} & \textbf{Simbolo} & \textbf{Comando} \\
	\hline
	$\cup$ & \textbackslash cup & $\cap$ & \textbackslash cap \\
	\hline
	$\subset$ & \textbackslash subset & $\supseteq$ & \textbackslash supseteq \\
	\hline
	$\leftarrow$ & \textbackslash leftarrow & $\rightarrow$ & \textbackslash rightarrow \\
	\hline
	$\vdash$ & \textbackslash vdash & $\vDash$ & \textbackslash vDash \\
	\hline
	$\wedge$ & \textbackslash wedge & $\vee$ & \textbackslash vee \\
	\hline

\end{tabular}

\section{Utilizzo di algorithm2e} %PARTE DEGLI ALGORITMI

per iniziare a scrivere un algoritmo devo utilizzare il comando \\ \textbackslash begin\{algorithm\}[H] e \textbackslash end\{algorithm\} alla fine del codice.
\\
con il comando \textbackslash SetAlgoLined posso decidere se mettere o no le linee per separare le righe dell'algoritmo.
\\
con il comando \textbackslash KwData posso scrivere i dati in ingresso dell'algoritmo.
\begin{algorithm}
	\SetAlgoLined
	\KwData{this text}
\end{algorithm}
\\
con il comando \textbackslash KwResult posso scrivere i dati in uscita dell'algoritmo.
\begin{algorithm}
	\SetAlgoLined
	\KwResult{how to write algorithm with \LaTeX2e }
\end{algorithm}
\newpage
\paragraph{For}con il comando \textbackslash For
\begin{algorithm}
	\For{content...}{for-block}
\end{algorithm}
\paragraph{ForEach}con il comando \textbackslash foreach
\begin{algorithm}
	\ForEach(content...){condition}{foreach-block}
\end{algorithm}


\paragraph{while} con il comando \textbackslash While 
\begin{algorithm}
	\While{content...}{while-block}
\end{algorithm}

\newpage 

\paragraph{If}con il comando \textbackslash If
\begin{algorithm}
	\If{content...}{then-block}
\end{algorithm}

\newpage

con il comando \textbackslash eIf posso scrivere un if-else. 
\\
\begin{algorithm}
	\eIf{condition}{then-block}{else-block}
\end{algorithm}
\\
\paragraph{Else}con il comando \textbackslash Else posso scrivere un else.
\\
\begin{algorithm}
	\eIf{Chiara è Piena}{si slaccia la cintura\; e si rompe il cazzo\; }{Si lamenta che ha fame}
\end{algorithm}



\end{document}
