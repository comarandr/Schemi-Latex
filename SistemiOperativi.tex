\documentclass{article}
\usepackage[utf8]{inputenc}
\usepackage[T1]{fontenc}
\usepackage{graphicx} % Required for inserting images
%before \begin
\usepackage{amssymb} %per i simboli matematici
\usepackage{mathtools} %per simboli 
\usepackage{color} %per colorare il codice
\usepackage{hyperref} %per i link
\DeclarePairedDelimiter\ceil{\lceil}{\rceil}
\DeclarePairedDelimiter\floor{\lfloor}{\rfloor}

\title{Sistemi Operativi - Teoria}
\author{Andrea Comar}
\date{\today}

\begin{document}
\maketitle
\newpage
\tableofcontents
\newpage

\part{Hardware e Architettura degli elaboratori}

\part{Introduzione ai Sistemi Operativi}
\section{Definizione}
Un sistema operativo è un programma che agisce come intermediario tra utente/programmatore e hardware. Coordina l'uso dell'hardware tra i vari programmi e utenti.
Il suo obbiettivo è quello di realizzare una \textit{macchina astratta} che implementi funzionalità di alto livello.
\begin{itemize}
    \item \textbf{assegnatore di risorse}: alloca le risorse in modo efficiente
    \item \textbf{Programma di controllo}: controlla l'esecuzione dei programmi per evitare errori
\end{itemize}

Le componenti di un sistema di calcolo sono:
\begin{itemize}
    \item \textbf{Hardware}: fornisce le risorse computazionali di base CPU, memoria, I/O
    \item \textbf{Sistema Operativo}: controlla e coordina le risorse hardware SO, driver, utility
    \item \textbf{Programmi di sistema}: programmi indipendenti dall'applicazione che forniscono servizi al SO (compilatori, editor,...)
    \item \textbf{Programmi applicativi}: programmi che definiscono il modo in cui le risorse del sistema sono usate per risolvere problemi computazionali dell'utente
    \item \textbf{Utenti}: persone, macchine, altri calcolatori.
\end{itemize}

\section{Storia}
\subsection{Esordi}
I primi calcolatori erano molto ingombranti 
e funzionavano unicamente da console, con un solo utente alla volta.
\subsection{II Generazione: Transistor  e sistemi batch} 

\subsection{III Generazione: Sistemi multiprogrammati}
\subsection{Sistemi time-sharing}

\part{Scheduling CPU}
\section{Introduzione}
L'introduzione della multiprogrammazione ha portato alla necessità di gestire l'allocazione della CPU tra i vari processi.
Ogni processo è caratterizzato da un \textbf{ciclo Burst CPU-I/O} che consiste in una sequenza di periodi
di esecuzione di CPU e attesa di I/O.






\end{document}

