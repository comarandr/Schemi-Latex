\documentclass{article}
\usepackage[utf8]{inputenc}
\usepackage{algorithm2e}
\usepackage{graphicx}

\title{Strumenti Utili}
\author{Andrea Comar}


\begin{document}
\maketitle
\section{Git}
\subsection{Termini}
\begin{itemize}
    \item \textbf{Repository}: Copia sincronizzabile della directory principale del progetto.
    \item \textbf{Commit}: è un'operazione che salva le modifiche fatte al repository. 
    \\
    Elenco delle modifiche effettuate a un progetto.
    \item \textbf{Master}: directory principale, raccoglie tutti i commit del progetto.
    \item \textbf{Branch}: indica una versione del tuo repository contentenente uno o più commit non presenti nel master.
    \item \textbf{Head}: la referenza all'ultimo commit di un ramo dilavoro a un branch al quale si sta lavorando
    \item \textbf{Merge}: è l'operazione che unisce due branch.
    \item \textbf{Pull Request}: è una richiesta di merge.
    \item \textbf{Fork}: è una copia di un repository.
    \item \textbf{Clone}: è una copia di un repository in locale.
    \item \textbf{Push}: è l'operazione che invia le modifiche fatte in locale al repository remoto.
    \item \textbf{Pull}: è l'operazione che scarica le modifiche fatte al repository remoto in locale.
    \item \textbf{Staging}: è l'area in cui si preparano i file da committare.
\end{itemize}
\subsection{Comandi}
\begin{itemize}
    \item \textbf{git init}: inizializza un repository vuoto.
    \item \textbf{git clone [url] [dir]}: clona un repository.
    \item \textbf{git add } aggiunge i file allo staging.
    \item \textbf{git commit}: salva le modifiche fatte allo staging.
    \item \textbf{git push}: invia le modifiche fatte in locale al repository remoto.
    \item \textbf{git pull}: scarica le modifiche fatte al repository remoto in locale.
    \item \textbf{git branch}: crea un nuovo branch.
    \item \textbf{git checkout}: cambia branch.
    \item \textbf{git merge}: unisce due branch.
    \item \textbf{git log}: mostra la lista dei commit.
    \item \textbf{git status}: mostra lo stato del repository.
    \item \textbf{git remote}: mostra i repository remoti.
    \item \textbf{git fetch}: scarica le modifiche fatte al repository remoto in locale.
    \item \textbf{git reset}: annulla le modifiche fatte allo staging.
    \item \textbf{git revert}: annulla un commit.
    \item \textbf{git rebase}: riscrive la storia del repository.
    \item \textbf{git tag}: aggiunge un tag a un commit.
    \item \textbf{git stash}: salva le modifiche fatte in locale.
    \item \textbf{git cherry-pick}: applica un commit a un branch.
    \item \textbf{git blame}: mostra chi ha modificato una riga di codice.
    \item \textbf{git config}: configura le impostazioni di Git.
    \item \textbf{git diff}: mostra le differenze tra i commit.
    \item \textbf{git rm}: rimuove i file dal repository.
    \item \textbf{git mv}: sposta o rinomina i file nel repository.
    \item \textbf{git show}: mostra i dettagli di un commit.
    \item \textbf{git archive}: crea un archivio dei file del repository.
    \item \textbf{git submodule}: gestisce i submodule nel repository.
    \item \textbf{git gc}: esegue la pulizia e l'ottimizzazione del repository.
    \item \textbf{git fsck}: verifica l'integrità del file system del repository.
    \item \textbf{git clean}: rimuove i file non tracciati dal repository.
\end{itemize}
\section{GitHub}

\end{document}